\chapter{Geänderter Quellcode}
\markboth{B. Geänderter Quellcode}{}
\label{ch:NeuQuellcode}

Im Folgenden wird ein Auszug geänderter Java Klassen aufgeführt. Der vollständige Quellcode befindet sich im Git-Repository von WBS Alarm als Release\footnote{\url{https://github.com/argh87/de.hs-fulda.et.it.project/releases/tag/1.1}} mit der Version 1.1.


\begin{lstlisting}[caption={Verarbeitung einer neuen Transaktion unter Verwendung von \code{TransaktionDao}. Die Klassen \code{TransaktionValidaton} und \code{TransaktionExecution} verwenden auch die neue Schnittstelle.}, label={lst:n_AddTransaktionActionImpl}]
package de.hsfulda.et.wbs.action.transaktion.impl;
import de.hsfulda.et.wbs.action.transaktion.AddTransaktionAction;
import de.hsfulda.et.wbs.core.WbsUser;
import de.hsfulda.et.wbs.core.data.TransaktionData;
import de.hsfulda.et.wbs.core.dto.TransaktionDto;
import org.springframework.stereotype.Component;
import javax.transaction.Transactional;
@Transactional
@Component
public class AddTransaktionActionImpl implements AddTransaktionAction {
  private final TransaktionValidaton validation;
  private final TransaktionExecution execution;
  private final TransaktionDao transaktionDao;
  private final TransaktionMailService transaktionMailService;
  public AddTransaktionActionImpl(TransaktionValidaton validation, TransaktionExecution execution,
      TransaktionMailService transaktionMailService, TransaktionDao transaktionDao) {
    this.validation = validation;
    this.execution = execution;
    this.transaktionMailService = transaktionMailService;
    this.transaktionDao = transaktionDao;
  }
  @Override
  public TransaktionData perform(WbsUser user, TransaktionDto dto) {
    return transaktionDao.hasAccessOnTransaktion(user, dto, () -> {
      validation.validateTransaktionDto(dto);
      TransaktionData transaktion = execution.createTransaktion(user, dto);
      transaktionMailService.sendMail(user, dto);
      return transaktion;
    });
  }
}
\end{lstlisting}

\begin{lstlisting}[caption={Schnittstelle für die Abhängigkeitsumkehr zu \code{repository}.}, label={lst:n_TransaktionDao}]
package de.hsfulda.et.wbs.action.transaktion.impl;
import de.hsfulda.et.wbs.core.WbsUser;
import de.hsfulda.et.wbs.core.data.BenutzerData;
import de.hsfulda.et.wbs.core.data.BestandData;
import de.hsfulda.et.wbs.core.data.GroesseData;
import de.hsfulda.et.wbs.core.data.TransaktionData;
import de.hsfulda.et.wbs.core.data.ZielortData;
import de.hsfulda.et.wbs.core.dto.TransaktionDto;
import de.hsfulda.et.wbs.entity.Benutzer;
import de.hsfulda.et.wbs.entity.Bestand;
import de.hsfulda.et.wbs.entity.Groesse;
import de.hsfulda.et.wbs.entity.Transaktion;
import de.hsfulda.et.wbs.entity.Zielort;
import org.springframework.data.domain.Page;
import org.springframework.data.domain.PageRequest;
import java.util.List;
import java.util.Optional;
import java.util.function.Supplier;
public interface TransaktionDao {
  <T> T hasAccessOnTraeger(WbsUser user, Long traegerId, Supplier<T> supplier);
  <T> T hasAccessOnTransaktion(WbsUser user, TransaktionDto dto, Supplier<T> supplier);
  <T> T hasAccessOnTransaktion(WbsUser user, Long transaktionId, Supplier<T> supplier);
  Optional<TransaktionData> getTransaktionData(Long id);
  Page<TransaktionData> getTransaktionPageByTraegerId(Long traegerId, PageRequest pageable);
  TransaktionData saveTransaktion(Transaktion transaktion);
  Optional<Long> findWareneingangByTraegerId(Long tragerId);
  Optional<Long> findLagerByTraegerId(Long tragerId);
  Benutzer getBenutzer(WbsUser user);
  List<BenutzerData> getEinkaeufer(Long benutzerId);
  ZielortData getZielortData(Long id);
  Zielort getZielort(Long id);
  boolean existsGroesse(Long groesseId);
  GroesseData getGroesseData(Long id);
  Groesse getGroesse(Long id);
  boolean existsBestand(Long id);
  BestandData getBestandData(Long bestandId);
  BestandData getBestandData(Long zielortId, Long groesseId);
  Optional<Bestand> getBestand(Long zielortId, Long groesseId);
  Bestand getBestand(Long bestandId);
  BestandData saveBestand(Bestand saved);
  void updateBestandAnzahl(Long bestandId, Long anzahl);
}
\end{lstlisting}

\begin{lstlisting}[caption={Implementierung der Schnittstelle \code{TransaktionsDao} auf Seite von \code{repository}.}, label={lst:n_TransaktionDaoImpl}]
package de.hsfulda.et.wbs.repository;
import de.hsfulda.et.wbs.action.transaktion.impl.TransaktionDao;
import de.hsfulda.et.wbs.core.WbsUser;
import de.hsfulda.et.wbs.core.data.BenutzerData;
import de.hsfulda.et.wbs.core.data.BestandData;
import de.hsfulda.et.wbs.core.data.GroesseData;
import de.hsfulda.et.wbs.core.data.TransaktionData;
import de.hsfulda.et.wbs.core.data.ZielortData;
import de.hsfulda.et.wbs.core.dto.TransaktionDto;
import de.hsfulda.et.wbs.core.exception.ResourceNotFoundException;
import de.hsfulda.et.wbs.core.exception.TransaktionValidationException;
import de.hsfulda.et.wbs.entity.Benutzer;
import de.hsfulda.et.wbs.entity.Bestand;
import de.hsfulda.et.wbs.entity.Groesse;
import de.hsfulda.et.wbs.entity.Transaktion;
import de.hsfulda.et.wbs.entity.Zielort;
import org.springframework.data.domain.Page;
import org.springframework.data.domain.PageRequest;
import org.springframework.stereotype.Component;
import javax.transaction.Transactional;
import java.util.List;
import java.util.Optional;
import java.util.function.Supplier;
@Transactional
@Component
public class TransaktionDaoImpl implements TransaktionDao {
  private final TransaktionRepository transaktionen;
  private final BenutzerRepository benutzer;
  private final GroesseRepository groessen;
  private final BestandRepository bestaende;
  private final ZielortRepository zielorte;
  private final AccessRepository access;
  public TransaktionDaoImpl(TransaktionRepository transaktionen, BenutzerRepository benutzer,
      GroesseRepository groessen, BestandRepository bestaende, ZielortRepository zielorte,
      AccessRepository access) {
    this.transaktionen = transaktionen;
    this.benutzer = benutzer;
    this.groessen = groessen;
    this.bestaende = bestaende;
    this.zielorte = zielorte;
    this.access = access;
  }
  @Override
  public <T> T hasAccessOnTraeger(WbsUser user, Long traegerId, Supplier<T> supplier) {
    Long counts = access.findTraegerByUserAndTraegerId(user.getId(), traegerId);
    return evaluteCount(counts, traegerId, "Träger", supplier);
  }
  @Override
  public <T> T hasAccessOnTransaktion(WbsUser user, TransaktionDto dto, Supplier<T> supplier) {
    checkCount(access.findTraegerByUserAndZielortId(user.getId(), dto.getVon()), dto.getVon(), "Zielort");
    checkCount(access.findTraegerByUserAndZielortId(user.getId(), dto.getNach()), dto.getNach(), "Zielort");
    dto.getPositions()
        .forEach(p -> {
          checkCount(access.findTraegerByUserAndGroesseId(user.getId(), p.getGroesse()), p.getGroesse(), "Gr" +
                         "öße");
        });
    return supplier.get();
  }
  @Override
  public <T> T hasAccessOnTransaktion(WbsUser user, Long transaktionId, Supplier<T> supplier) {
    Long counts = access.findTraegerByUserAndTransaktionId(user.getId(), transaktionId);
    return evaluteCount(counts, transaktionId, "Transaktion", supplier);
  }
  private void checkCount(final Long counts, final Long id, final String resource) {
    if (counts == 0) {
      throw new ResourceNotFoundException("{1} mit ID {0} nicht gefunden.", id, resource);
    }
  }
  private <T> T evaluteCount(final Long counts, final Long id, final String resource, final Supplier<T> supplier) {
    checkCount(counts, id, resource);
    return supplier.get();
  }
  @Override
  public Optional<TransaktionData> getTransaktionData(Long id) {
    return transaktionen.findByIdAsData(id);
  }
  @Override
  public Page<TransaktionData> getTransaktionPageByTraegerId(Long traegerId, PageRequest pageable) {
    return transaktionen.findAllAsDataByTraegerId(traegerId, pageable);
  }
  @Override
  public TransaktionData saveTransaktion(Transaktion transaktion) {
    return transaktionen.save(transaktion);
  }
  @Override
  public Optional<Long> findWareneingangByTraegerId(Long tragerId) {
    return zielorte.findWareneingangByTraegerId(tragerId);
  }
  @Override
  public Optional<Long> findLagerByTraegerId(Long tragerId) {
    return zielorte.findLagerByTraegerId(tragerId);
  }
  @Override
  public Benutzer getBenutzer(WbsUser user) {
    Optional<Benutzer> managed = benutzer.findById(user.getId());
    return managed.orElseThrow(
        () -> new ResourceNotFoundException("Benutzer mit ID {0} nicht gefunden.", user.getId()));
  }
  @Override
  public List<BenutzerData> getEinkaeufer(Long benutzerId) {
    return benutzer.findAllEinkaeuferByUserId(benutzerId);
  }
  @Override
  public ZielortData getZielortData(Long id) {
    Optional<ZielortData> managed = findZielortByIdAndAktivIsTrue(id);
    return managed.orElseThrow(() -> new ResourceNotFoundException("Zielort mit ID {0} nicht gefunden.", id));
  }
  @Override
  public Zielort getZielort(Long id) {
    Optional<Zielort> managed = zielorte.findById(id);
    return managed.orElseThrow(() -> new ResourceNotFoundException("Zielort mit ID {0} nicht gefunden.", id));
  }
  private Optional<ZielortData> findZielortByIdAndAktivIsTrue(Long id) {
    return zielorte.findByIdAndAktivIsTrue(id);
  }
  @Override
  public boolean existsGroesse(Long groesseId) {
    return groessen.existsById(groesseId);
  }
  @Override
  public GroesseData getGroesseData(Long id) {
    Optional<GroesseData> managed = findGroesseByIdAndAktivIsTrue(id);
    return managed.orElseThrow(() -> new ResourceNotFoundException("Groesse mit ID {0} nicht gefunden.", id));
  }
  @Override
  public Groesse getGroesse(Long id) {
    Optional<Groesse> managed = groessen.findById(id);
    return managed.orElseThrow(() -> new ResourceNotFoundException("Groesse mit ID {0} nicht gefunden.", id));
  }
  private Optional<GroesseData> findGroesseByIdAndAktivIsTrue(Long id) {
    return groessen.findByIdAndAktivIsTrue(id);
  }
  @Override
  public boolean existsBestand(Long id) {
    return bestaende.existsById(id);
  }
  @Override
  public BestandData getBestandData(Long bestandId) {
    Optional<BestandData> managed = bestaende.findByIdAsData(bestandId);
    return managed.orElseThrow(
        () -> new ResourceNotFoundException("Bestand mit ID {0} nicht gefunden.", bestandId));
  }
  @Override
  public BestandData getBestandData(Long zielortId, Long groesseId) {
    Optional<BestandData> managed = findBestandByZielortIdAndGroesseId(zielortId, groesseId);
    return managed.orElseThrow(
        () -> new TransaktionValidationException("Für den Zielort {0} wurde kein Bestand gefunden (Größe {1}).",
            zielortId, groesseId));
  }
  @Override
  public Optional<Bestand> getBestand(Long zielortId, Long groesseId) {
    return bestaende.findByZielortIdAndGroesseId(zielortId, groesseId);
  }
  @Override
  public Bestand getBestand(Long bestandId) {
    Optional<Bestand> managed = bestaende.findById(bestandId);
    return managed.orElseThrow(
        () -> new ResourceNotFoundException("Bestand mit ID {0} nicht gefunden.", bestandId));
  }
  private Optional<BestandData> findBestandByZielortIdAndGroesseId(Long zielortId, Long groesseId) {
    return bestaende.findByZielortIdAndGroesseIdAsData(zielortId, groesseId);
  }
  @Override
  public BestandData saveBestand(Bestand bestand) {
    return bestaende.save(bestand);
  }
  @Override
  public void updateBestandAnzahl(Long bestandId, Long anzahl) {
    bestaende.updateAnzahl(bestandId, anzahl);
  }
}
\end{lstlisting}

\begin{lstlisting}[caption={Schnittstelle für die neue Komponente \code{mail}.}, label={lst:n_MailService}]
package de.hsfulda.et.wbs.action;
import de.hsfulda.et.wbs.core.exception.MailConnectionException;
public interface MailService { 
  void send(Mail mail) throws MailConnectionException;
}
\end{lstlisting}

\begin{lstlisting}[caption={Typ von Objekt, das an die E-Mail Schnittstelle übergeben werden kann.}, label={lst:n_Mail}]
package de.hsfulda.et.wbs.action;
public interface Mail {
  String getSubject();
  String[] getTo();
  String getText();
}
\end{lstlisting}

\begin{lstlisting}[caption={Implementierung der Schnittstelle \code{MailService} in der Komponente \code{mail}.}, label={lst:n_MailServiceImpl}]
package de.hsfulda.et.wbs.mail;
import de.hsfulda.et.wbs.action.Mail;
import de.hsfulda.et.wbs.action.MailService;
import de.hsfulda.et.wbs.core.exception.MailConnectionException;
import de.hsfulda.et.wbs.core.exception.MailDeliveryException;
import org.slf4j.Logger;
import org.slf4j.LoggerFactory;
import org.springframework.beans.factory.annotation.Value;
import org.springframework.mail.MailSendException;
import org.springframework.mail.javamail.JavaMailSender;
import org.springframework.mail.javamail.MimeMessageHelper;
import org.springframework.stereotype.Service;
import javax.mail.MessagingException;
import javax.mail.internet.InternetAddress;
import javax.mail.internet.MimeMessage;
import java.io.UnsupportedEncodingException;
@Service
public class MailServiceImpl implements MailService {
  private static final Logger LOGGER = LoggerFactory.getLogger(MailServiceImpl.class);
  @Value("${wbs.mail.active}")
  private boolean active;
  @Value("${spring.mail.username}")
  private String fromMailAdress;
  @Value("${wbs.mail.personal}")
  private String fromMailPersonal;
  private final JavaMailSender mailSender;
  public MailServiceImpl(JavaMailSender mailSender) {
    this.mailSender = mailSender;
  }
  @Override
  public void send(Mail mail) throws MailConnectionException {
    if (!active) {
      return;
    }
    try {
      MimeMessage message = mailSender.createMimeMessage();
      MimeMessageHelper helper = new MimeMessageHelper(message);
      helper.setSubject(mail.getSubject());
      helper.setFrom(new InternetAddress(fromMailAdress, fromMailPersonal));
      helper.setTo(mail.getTo());
      helper.setText(mail.getText());
      LOGGER.debug("Sending E-Mail \"{}\" to {} ", mail.getSubject(), String.join(", ", mail.getTo()));
      mailSender.send(message);
      LOGGER.info("Send E-Mail \"{}\" to {} ", mail.getSubject(), String.join(", ", mail.getTo()));
    } catch (MessagingException | UnsupportedEncodingException e) {
      throw new MailDeliveryException(e, "Beim Senden der Mail {0} ist ein Fehler aufgetreten", mail.toString());
    } catch (MailSendException e) {
      throw new MailConnectionException();
    }
  }
}
\end{lstlisting}
