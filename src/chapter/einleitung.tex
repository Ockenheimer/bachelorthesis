\chapter{Einleitung}
\markboth{Einleitung}{}
\label{ch:Einleitung}

Während des Softwareprojekts im Rahmen des Studiums der Sozialinformatik wurde eine Open Source Software namens WBS Alarm entwickelt, welche die Kleiderkammer der freiwilligen Feuerwehr in Eschenstruth digitalisieren sollte. Das Projekt erfolgte in vier Phasen á zwei Monaten, die sich insgesamt über zwei Jahre hinweg verteilten. In der ersten Phase wurden die Anforderungen aufgenommen und das Vorgehen geplant. Dabei wurde auch festgelegt, dass es sich um eine Webanwendung handeln soll, damit die Mitglieder der Feuerwehr auch mobil arbeiten können. In der zweiten Phase wurde die Administration erstellt, in der alle notwendigen Daten erfasst werden können, die für eine Kleiderbuchung benötigt werden. Zudem wurde das System gegen unbefugten Zugriff mit einem Login gesichert. In der dritten Phase wurde die zentrale Geschäftslogik entwickelt, bei der Transaktionen zwischen Ortsteilen gebucht werden können. Dabei wurde die Software auch auf einem Server für die Kunden bereitgestellt und veröffentlicht. In der vierten Phase wurden Berichte hinzugefügt, ein Versand von Statusnachrichten per E-Mail eingerichtet und die Struktur der Oberfläche angepasst.

In der dritten Phase sind einige Teile nicht zukunftsfähig entwickelt worden. Die Anforderungen haben mehr Zeit in Anspruch genommen als ursprünglich dafür geplant gewesen war, weshalb die Bearbeitungszeit für andere Teilaspekte des Projekts gekürzt werden musste. Dies betraf insbesondere die Transaktionserzeugung als zentralsten und komplexesten Teil der Geschäftslogik. 

In dieser Arbeit wird zunächst geprüft, inwieweit eine saubere Softwarearchitektur dabei helfen kann die Transaktionserzeugung wartbar und erweiterbar zu halten bzw. sie zu verbessern und diese Verbesserungen anschließend umgesetzt. 

In Kapitel~\ref{ch:Grundlagen} wird zunächst das Projekt WBS Alarm selbst vorgestellt und die Anforderungen an die Transaktionserzeugung definiert und erläutert. Zudem wird auf zwei Fallstudien Bezug genommen, in denen beschrieben wird, warum eine saubere Softwarearchitektur benötigt wird und wie dies für Weiterentwicklungen eine Rolle spielt. Danach werden Design- und Komponentenprinzipien, sowie ein Ansatz einer sauberen Softwarearchitektur aus aktueller Fachliteratur zusammengefasst, die für die Optimierung der Transaktionserfassung von Bedeutung sind.

Die zentrale Fragestellung lautet in dieser Arbeit: Kann anhand der Design- und Komponentenprinzipien die Struktur der Transaktionserfassung optimiert werden, sodass bei Anforderungsänderungen oder ‑erweiterungen diese effizient angepasst werden kann? In Kapitel~\ref{ch:Fragestellung} wird hierauf genauer eingegangen.

Im Kapitel~\ref{ch:Vorgehensweise} wird mit Hilfe von UML der Ausgangszustand modelliert. Dieser wird daraufhin anhand der erarbeiteten Grundlagen analysiert. Wenn sich in der Analyse ergeben sollte, dass gegen bestimmte Design- und Komponentenprinzipien verstoßen wurde, wird eine Anpassung erst mit UML geplant und dann entsprechend mit Java umgesetzt.

Die Erkenntnisse werden im Kapitel~\ref{ch:Auswertung} ausgewertet.

Schließlich wird in Kapitel~\ref{ch:Fazit} ein Fazit über die genannten Prinzipien gezogen und ob sich eine Umstrukturierung für das Projekt WBS Alarm gelohnt hat.