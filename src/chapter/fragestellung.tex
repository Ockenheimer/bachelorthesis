\chapter{Fragestellung}
\markboth{Fragestellung}{}
\label{ch:Fragestellung}

Schlecht designter und komplizierter Code führt zu schlechterer Erweiterbarkeit, aufwändigerer Wartung und es wird mehr Zeit benötigt neue Funktionen und Anforderungen umzusetzen. Im schlimmsten Fall kann ein System nicht weiterentwickelt werden und muss von Grund auf neu erstellt oder durchdacht werden. Dies kostet Zeit und Ressourcen.

Daher stellen sich Fragen um den aktuellen Zustand der Transaktionserfassung in WBS Alarm. Sind Komponenten nach den \oge Design"= und Komponentenprinzipien aufgebaut? Wurden zwischen den Komponenten Grenzen richtig gezogen? An welchen Stellen wird gegen die Vorgaben verstoßen und wie kann dies gelöst werden?

Hierfür wird der aktuelle Stand von WBS Alarm analysiert, indem die Komponenten \bzgl der Transaktionserfassung aufgelistet und mittels UML in Zusammenhang gebracht werden. Die Komponenten werden weiterhin auf deren Grenzüberschreitung analysiert. Hierzu wird zudem für die einzelnen Komponenten die Abstraktheit $A$ und die Instabilität $I$ gemessen.
